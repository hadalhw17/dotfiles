\documentclass[a4paper,12pt]{article}
\usepackage{graphicx} % This package is used to allow us to include a
                      % wide range of graphics formats within the
                      % document.  We can add many different packages
                      % with LaTeX, but for most purposes, a small
                      % subset is sufficient.
\usepackage{hyperref}
\usepackage{listings}
\begin{document}

Hello world! What chance is there of this working?

\section{Introduction}
\label{mystart} % Note that the percent character is used for comments - everything that follows on the same line is ignored.
\textbf{Computer Vision} - is an area of science that studies artifitial inteligent, in particular
robotics and all technologies that are used to obtain images of real world objects and then apply some actions on
them without(full or partial) human presence. 

In the assignment given I had to study a small subset of \textbf{CIFAR-10} dataset. \textbf{CIFAR-10} dataset
consists of 60000 32x32 rgb color images, where all of the images are separated in 10 classes. Original dataset has 
50000 training images and 10000 test images, and each class has 6000 images. The dataset given to us has 1000 training
and 100 testing images. There are a total of 10000 training images and 1000 testng images in the dataset.

After analyzing the dataset I had to make a program that classifies images from the dataset provided. I used several 
methods in order to achieve the given task. Perform classification using k-nearest neighbour, classification using
Support Vetor Machine, clustering using k-means classifier and neural network. Redcuce data dimentionality using 
Linear Discriminant Analysis and perform steps above once again on reduced data. To code all of the techniques above 
I used \textit{scikit-learn} and \textit{keras} libraries. 

\section{Method}
In general the given task can be splitted onto several easier tasks, which are:
\begin{enumerate}
  \item Feature extraction
  \item Data preparation
  \item Classification
  \item Result analysis
\end{enumerate}
Each of the tasks above steps above can be easily be completed with the use of \textit{scikit-learn} 
and \textit{keras} libraries. 

\subsection{Feature extaction}
In my opinion operating with raw pixel data in anything but neural network will be a bad idea, since 
it does not hold any important information for classifiers like k-means or k-nearest neighbour, but color.
Color is not a reliable feature for classification, because same object (for example plane) can be coloured differently
on different pictures (for example planes that are being operated by S7 airlines are mostly green, while planes operated
by Aeroflot airlines are silver). 
In order to perform hight quality classification I am going to use \textbf{Histogram of Oriented Gradients (HOG)} features. 
In order to compute \textbf{HOG},
\begin{enumerate}
  \item Make image black and white, since information about colors will be useless.
  \item Take a pixel, we need to look at the pixels surrounding it.
  \item Determine how dark is our pixel compared to its adjasent pixels
  \item Draw an arrow that represents the direction in which pixels become darker.
  \item Repeat from step 2 for each pixel in the image.
\end{enumerate}
This method has several key advantages over other descriptors is that it works with local grid and
is not sensitive to geometric or photometric properties of the object, appart from its orientation. 
This is beneficial objects with large area.

\subsection{Data preparation}
During this step 2 things could possibly be done:
\begin{itemize}
  \item Perform dimentionality reduction
  \item Reshape data for neural network.
\end{itemize}
As a method for dimentionality reduction I decided to choose 
\textbf{Linear Discriminant Analysis(LDA)},
since it provides better between class separation than 
\textbf{Principal Component Analysis(PCA)}. 

\textbf{LDA} is a method that is used in statistics, machine learning,
object recocnition in order to find linear combination of properties.
The combination obtained can be used as a linear classifier, or more frequently
as a metod of dimentionality reduction before classification.

By reshaping data before feeding it to neural network I mean generating binary class
matrix of labels using:
\begin{lstlisting}
  keras.utils.to_categorical(y,num_classes=None)
\end{lstlisting}
\end{document}



